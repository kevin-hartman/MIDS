%%%%%%%%%%%%%%%%%%%%%%%%%%%%%%%%%%%%%%%%%%%%%%%%%%%%%%%%%%%%%%%%%%%%%%%%%%%%
% Statistics for Data Science (DataSci w203)
% Unit 4 part 2 homework
%%%%%%%%%%%%%%%%%%%%%%%%%%%%%%%%%%%%%%%%%%%%%%%%%%%%%%%%%%%%%%%%%%%%%%%%%%%%

% Setup
\documentclass[12pt,a4paper]{article}
\usepackage[inner=1.5cm,outer=1.5cm,top=2.5cm,bottom=2.5cm]{geometry}
\usepackage{graphicx, subfig}
\usepackage[english]{babel}
\usepackage{amsmath}
\usepackage{amssymb}
\numberwithin{equation}{subsection}
\usepackage{hyperref}
\usepackage{listings}


\usepackage{Sweave}
\begin{document}
\Sconcordance{concordance:Kevin_Hartman_unit_4_part_2_hw.tex:Kevin_Hartman_unit_4_part_2_hw.Rnw:%
1 130 1}

\Sconcordance{concordance:Kevin_Hartman_unit_4_part_2_hw.tex:Kevin_Hartman_unit_4_part_2_hw.Rnw:%
1 130 1}


\author{Kevin Hartman (W203 Wednesday 6:30pm Summer 2019)}
\date{6/5/2019}
\title{Statistics for Data Science \\
       Unit 4 Part 2 Homework: Continuous Random Variables}
\maketitle

%----------------------------------------------------------------------------------------
%----------------------------------------------------------------------------------------
\begin{enumerate}

\item \textbf{Processing Pasta}

A certain manufacturing process creates pieces of pasta that vary by length.  Suppose that the length of a particular piece, $L$, is a continuous random variable with the following probability density function.

$$f(l) = \begin{cases} 0, &l \leq 0 \\
l/2, &0 < l \leq 2 \\ 
0, &2 < l
\end{cases}
$$

\begin{enumerate}
\item Write down a complete expression for the cumulative probability function of $L$.
\\
\textbf{\underline{Answer:}}
\\
The CDF of L is defined as
$$F(l) = \int_{y=-\infty}^{l} f(y) dy$$


When $l \leq 0$,
$$F(l) = \int_{y=-\infty}^{0}  0 \cdot dy = 0$$

When $0 < l < 2$,

$$F(l) = \int_{y=-\infty}^{0} 0 \cdot dy + \int_{y=0}^{l} \frac{y}{2} dy = 0 + \frac{y^2}{4}\bigg|_0^l= \frac{l^2}{4}$$

When $l \geq 2$,

$$F(l) = \int_{y=-\infty}^{0} 0 \cdot dy + \int_{y=0}^{2} \frac{y}{2} dy  + \int_{y=2}^{\infty} 0 \cdot dy = 0 + \frac{y^2}{4}\bigg|_0^2 + 0 = \frac{2^2}{4} = 1$$

Which gives us

$$F(l) = \begin{cases}
0, & l \leq 0 \\
\frac{l^{2}}{4}, & 0 < l < 2\\
1, & 2 \leq l \\
\end{cases}$$

\item Using the definition of expectation for a continuous random variable, compute the expected length of the pasta, $E(L)$.
\\
\textbf{\underline{Answer:}}
\\
$$E(L) = \int_{-\infty}^{\infty} l \cdot f(l) dl$$
Assembling from the different ranges:
$$E(L) = \int_{-\infty}^{0} l \cdot 0 dl + \int_{0}^{2} l \cdot \frac{l}{2} dl + \int_{2}^{\infty} l \cdot 0 dl $$
$$ = \int_{0}^{2} \frac{l^2}{2} dl = \frac{l^3}{6}\bigg|_0^2= \frac{8}{6} $$
\end{enumerate}

\item \textbf{The Warranty is Worth It}

Suppose the life span of a particular (shoddy) server is a continuous random variable, T, with a uniform probability distribution between 0 and 1 year.  The server comes with a contract that guarantees you money if the server lasts less than 1 year.  In particular, if the server lasts $t$ years, the manufacturer will pay you $g(t)= \$100(1-t)^{1/2}$.  Let $X = g(T)$ be the random variable representing the payout from the contract.

Compute the expected payout from the contract, $E(X) = E(g(T))$.
\\
\textbf{\underline{Answer:}}
\\
$$ E(X) = E(g(T)) = \int g(t)f(t) dt  =  \int 100 (1 - t)^\frac{1}{2} f(t)dt$$

Since T has a uniform probability distribution between 0 and 1, $f(t)=\frac{1}{(1-0)} = 1$
$$E(X) = \int_{t=0}^1 100 (1 - t)^\frac{1}{2}dt = \frac{200}{3}(1-t)^{\frac{3}{2}} \bigg|_0^1=\frac{200}{3}=\$66.67$$

\item \textbf{(Lecture)\#Fail}

Suppose the length of Paul Laskowski's lecture in minutes is a continuous random variable C, with pmf $f(t) = e^{-t}$ for $t > 0$.  This is an example of an exponential random variable, and it has some special properties.  For example, suppose you have already sat through t minutes of the lecture, and are interested in whether the lecture is about to end immediately.  In statistics, this can be represented by something called the \textit{hazard rate}:

$$h(t) = \frac{f(t)}{1-F(t)}$$

To understand the hazard rate, think of the numerator as the probability the lecture ends between time $t$ and time $t+ dt$.  The denominator is just the probability the lecture does not end before time $t$.  So you can think of the fraction as the conditional probability that the lecture ends between $t$ and $t + dt$ given that it did not end before $t$.

Compute the hazard rate for C.
\\
\textbf{\underline{Givens:}}
\\
The hazard rate is defined as
$$h(t) = \frac{f(t)}{1-F(t)}$$

The CDF of C, where $t$ is the length of C in minutes, is defined as
$$F(t) = \int_{y=-\infty}^{t} f(y) dy$$

The PMF of C, where $t$ is the length of C in minutes, is defined as
$$f(t) = \begin{cases} 0, &t \leq 0 \\
e^{-t}, &0 < t
\end{cases}
$$
\\
\textbf{\underline{Answer:}}
\\
When $t \leq 0$,
$$F(t) = \int_{y=-\infty}^{0}  0 \cdot dy = 0$$

When $0 < t$
$$F(t) = \int_{y=0}^{t} e^{-y} dy = (-e^y + B)\bigg|_0^t = (-e^t + B) - (-e^0 + B) = -e^{-t} + B + 1 - B = 1 - e^{-t}$$

When combined, $F(t) = 1 - e^{-t}$

Replacing $f(t)$ and $F(t)$ in the hazard rate for C,

$$ h_C(t) = \frac{e^{-t}}{1 - (1 - e^{-t})} = \frac{e^{-t}}{e^{-t}} = 1 \text{ for } t > 0$$

\item \textbf{Optional Advanced Exercise: Characterizing a Function of a Random Variable}

Let $X$ be a continuous random variable with probability density function $f(x)$, and let $h$ be an invertible function where $h^{-1}$ is differentiable.  Recall that $Y = h(X)$ is itself a continuous random variable.  Prove that the probability density function of $Y$ is 

$$g(y) =f(h^{-1}(y)) \cdot \left| \frac{d}{dy}h^{-1}(y) \right| $$
\\
\textbf{\underline{Proof:}}
\\
If $h$ is invertible and $Y = h(X)$ then $X = h^{-1}(Y)$

If $h^{-1}$ is differentiable then so is $h$, and $h$ will always be increasing or decreasing.

Let $h$ be increasing, as proof for the other side would be similar.

Following the above,

$$\text{    if }Y = h(X) \text{   then } P(Y \leq y) = P(h(X) \leq y)$$

$$\text{and if } X = h^{-1}(Y)  \text{ then } P(Y \leq y) = P(X \leq h^{-1}(y)$$

Thus, the CDF  of Y, G(y), is
$$ G(y) = \int_{-\infty}^{h^{-1}(y)}f(x)dx = \int_{-\infty}^{y}f(h^{-1}(u)) \cdot \frac{d}{dy}h^{-1}(u)du \quad \text{ by substitution }$$
$$ = \int_{-\infty}^{y}f(h^{-1}(u)) \cdot \bigg|\frac{d}{dy}h^{-1}(u)\bigg|du \quad \text{ since h is increasing }$$

Since $g(y) = G^{'}(y)$,
$$g(y) = f(h^{-1}(y)) \cdot \bigg|\frac{d}{dy}h^{-1}(y)\bigg|$$


\end{enumerate}

\end{document}
