%%%%%%%%%%%%%%%%%%%%%%%%%%%%%%%%%%%%%%%%%%%%%%%%%%%%%%%%%%%%%%%%%%%%%%%%%%%%
% Statistics for Data Science (DataSci w203)
% Unit 4 homework
%%%%%%%%%%%%%%%%%%%%%%%%%%%%%%%%%%%%%%%%%%%%%%%%%%%%%%%%%%%%%%%%%%%%%%%%%%%%

% Setup
\documentclass[12pt,a4paper]{article}
\usepackage[inner=1.5cm,outer=1.5cm,top=2.5cm,bottom=2.5cm]{geometry}
\usepackage{graphicx}
\usepackage[english]{babel}
\usepackage{amsmath}
\usepackage{amssymb}
\numberwithin{equation}{subsection}
\usepackage{hyperref}

%%%%%%%%%%%%%%%%%%%%%%%%%%%%%%%%%%%%%%%%%%%%%%%%%%%%%%%%%%%%%%%%%%%%%%%%%%%%


\begin{document}

\title{Statistics for Data Science \\
       Unit 4 Part 2 Homework: Continuous Random Variables}
%\institute{Institution}
\maketitle

%----------------------------------------------------------------------------------------
%----------------------------------------------------------------------------------------
\begin{enumerate}

\item \textbf{Processing Pasta}

A certain manufacturing process creates pieces of pasta that vary by length.  Suppose that the length of a particular piece, $L$, is a continuous random variable with the following probability density function.

$$f(l) = \begin{cases} 0, &l \leq 0 \\
l/2, &0 < l \leq 2 \\ 
0, &2 < l
\end{cases}
$$

\begin{enumerate}
\item Write down a complete expression for the cumulative probability function of $L$.
\item Using the definition of expectation for a continuous random variable, compute the expected length of the pasta, $E(L)$.
\end{enumerate}

\item \textbf{The Warranty is Worth It}

Suppose the life span of a particular (shoddy) server is a continuous random variable, T, with a uniform probability distribution between 0 and 1 year.  The server comes with a contract that guarantees you money if the server lasts less than 1 year.  In particular, if the server lasts $t$ years, the manufacturer will pay you $g(t)= \$100(1-t)^{1/2}$.  Let $X = g(T)$ be the random variable representing the payout from the contract.

Compute the expected payout from the contract, $E(X) = E(g(T))$.

\item \textbf{(Lecture)\#Fail}

Suppose the length of Paul Laskowski's lecture in minutes is a continuous random variable C, with pmf $f(t) = e^{-t}$ for $t > 0$.  This is an example of an exponential random variable, and it has some special properties.  For example, suppose you have already sat through t minutes of the lecture, and are interested in whether the lecture is about to end immediately.  In statistics, this can be represented by something called the \textit{hazard rate}:

$$h(t) = \frac{f(t)}{1-F(t)}$$

To understand the hazard rate, think of the numerator as the probability the lecture ends between time $t$ and time $t+ dt$.  The denominator is just the probability the lecture does not end before time $t$.  So you can think of the fraction as the conditional probability that the lecture ends between $t$ and $t + dt$ given that it did not end before $t$.

Compute the hazard rate for C.

\item \textbf{Optional Advanced Exercise: Characterizing a Function of a Random Variable}

Let $X$ be a continuous random variable with probability density function $f(x)$, and let $h$ be an invertible function where $h^{-1}$ is differentiable.  Recall that $Y = h(X)$ is itself a continuous random variable.  Prove that the probability density function of $Y$ is 

$$g(y) =f(h^{-1}(y)) \cdot \left| \frac{d}{dy}h^{-1}(y) \right| $$

\end{enumerate}

\end{document}