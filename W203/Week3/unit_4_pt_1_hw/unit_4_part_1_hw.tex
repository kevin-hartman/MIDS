%%%%%%%%%%%%%%%%%%%%%%%%%%%%%%%%%%%%%%%%%%%%%%%%%%%%%%%%%%%%%%%%%%%%%%%%%%%%
% Statistics for Data Science (DataSci w203)
% Unit 4 part 1 homework
%%%%%%%%%%%%%%%%%%%%%%%%%%%%%%%%%%%%%%%%%%%%%%%%%%%%%%%%%%%%%%%%%%%%%%%%%%%%

% Setup
\documentclass[12pt,a4paper]{article}
\usepackage[inner=1.5cm,outer=1.5cm,top=2.5cm,bottom=2.5cm]{geometry}
\usepackage{graphicx}
\usepackage[english]{babel}
\usepackage{amsmath}
\usepackage{amssymb}
\numberwithin{equation}{subsection}
\usepackage{hyperref}

%%%%%%%%%%%%%%%%%%%%%%%%%%%%%%%%%%%%%%%%%%%%%%%%%%%%%%%%%%%%%%%%%%%%%%%%%%%%


\begin{document}

\title{Statistics for Data Science \\
       Unit 4 Part 1 Homework: Discrete Random Variables}
\maketitle

%----------------------------------------------------------------------------------------
%----------------------------------------------------------------------------------------
\begin{enumerate}

\item \textbf{Best Game in the Casino}߆

You flip a fair coin 3 times, and get a different amount of money depending on how many heads you get. For 0 heads, you get \$0. For 1 head, you get \$2. For 2 heads, you get \$4. Your expected winnings from the game are \$6. 

\begin{enumerate}
\item How much do you get paid if the coin comes up heads 3 times?
\item Write down a complete expression for the cumulative probability function for your winnings from the game.
\end{enumerate}

\item \textbf{Reciprocal Dice}

Let $X$ be a random variable representing the outcome of rolling a 6-sided die.  Before the die is rolled, you are given two options:

\begin{enumerate}
\item You get $1/E(X)$ in dollars right away.
\item You wait until the die is rolled, then get $1/X$ in dollars.
\item Which option is better for you, in expectation?
\end{enumerate}



\item \textbf{The Baseline for Measuring Deviations}

Given any random variable $X$ and a real number $t$, we can define another random variable $Y = (X - t)^2$. In other words, for any random variable $X$, we can choose a real number, $t$, as a baseline and calculate the squared deviation of $X$ away from $t$.

You might wonder why we often square deviations (instead of taking an absolute value, or cubing them, etc.).  This exercise will shed some light on why this is a natural choice.

\begin{enumerate}
\item Write down an expression for $E(Y)$ and simplify it as much as you can.  Even though we haven't proved this yet, you can use the fact that for any two random variables, $A$ and $B$, $E(A + B) = E(A) + E(B)$.
\item Taking a partial derivative with respect to $t$, compute the value of $t$ that minimizes $E(Y)$.  (Hint: Your answer should be a very familiar value)
\item What is the value of $E(Y)$ for this choice of $t$?
(Hint: this should also be a very familiar value)
\end{enumerate}

\item \textbf{Optional Advanced Exercise: Heavy Tails}

One reason to study the mathematical foundation of statistics is to recognize situations where common intuition can break down.  An unusual class of distributions are those we call \textit{heavy-tailed}.  The exact definition varies, but we'll say that a heavy-tailed distribution is one for which not all moments are finite.  Consider a random variable $M$ with the following pmf:

$$p_M(x) = \begin{cases}
c/x^3, &x \in \{1,2,3,...\}\\
0, &otherwise.\\
\end{cases}
$$

where $c$ is a constant (you can calculate its value if you like, but it's not important).

\begin{enumerate}
\item Is $E(M)$ finite?
\item Is $V(M)$ finite?
\end{enumerate}

Heavy-tailed distributions may seem odd, but they're not as rare as you might suspect.  Researchers argue that the distribution of wealth is heavy-tailed; so is the distribution of computer file sizes, insurance payouts, and area burned by forest fires.  These random variables are problematic in that a lot of common statistical techniques don't work on them.  For this class, we'll assume that all of our variables don't have heavy-tails.

\end{enumerate}

\end{document}