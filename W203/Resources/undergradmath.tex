\RequirePackage{mmap}  % make PDF copy and paste-able
\documentclass[twocolumn]{article}
\usepackage{amsmath, amsthm, amssymb}
\usepackage[utf8]{inputenc}
\usepackage[T1]{fontenc}
% \usepackage{lmodern} % from http://tex.stackexchange.com/a/115089/121234
\usepackage[margin=0.5in]{geometry}

\setlength{\parskip}{0ex}
\usepackage{ragged2e}
  \setlength{\RaggedRightParindent}{1em}

\usepackage{url}
\usepackage{listings}
\lstset{%
  basicstyle=\scriptsize\ttfamily,  % the size of the fonts 
  columns=fixed,          % anything else is horrifying
  showspaces=false,       % show spaces using underscores?
  showstringspaces=false, % underline spaces within strings?
  showtabs=false,         % show tabs within strings?
  xleftmargin=1.5em,      % left margin space
}
\lstdefinestyle{inline}{basicstyle=\ttfamily}

\usepackage[dvipsnames]{xcolor}
  \definecolor{headcolor}{HTML}{E34234}  % vermillion
\usepackage{titlesec}
% \titleformat{ command }[ shape ]{ format }{ label }{ sep }{ before-code }[ after-code ]
% \titlespacing*{ command }{ left }{ before-sep }{ after-sep }[ right-sep ]
\titleformat{\section}[runin]{\color{headcolor}\bf}{}{0em}{}
  \titlespacing*{\section}{0em}{0.65ex}{0.67em}

% TODO


\pagestyle{empty}
\begin{document}\thispagestyle{empty}
\RaggedRight
\begin{center}
  {\large\color{headcolor}\textbf{\LaTeX{} Math for Undergrads}}  \\
\end{center}

\section{Rule One}
Any mathematics at all, 
even a single character, goes in a mathematical setting.
Thus, for ``the value of \( x \) is \( 7 \)\,'' enter
`\lstinline[style=inline]!the value of \( x \) is \( 7 \)!'.

\section{Template}
Your document should contain at least this.
\begin{lstlisting}
\documentclass{article}
\usepackage{amsmath, amssymb, amsthm}
\usepackage[utf8]{inputenc}

\begin{document}
  --document body here--
\end{document}     
\end{lstlisting}


\section{Common constructs}
\begin{center}
  \begin{tabular}{@{} *{1}{l@{\hspace{1.5em}}} l @{}}
    \(x^2\)\quad\lstinline[style=inline]!x^2!
      &\(\sqrt{2}\), \(\sqrt[n]{3}\)\quad\lstinline[style=inline]!\sqrt{2}!,
        \lstinline[style=inline]!\sqrt[n]{3}!                           \\     
    \(x_{i,j}\)\quad\lstinline[style=inline]!x_{i,j}!
      &\(\frac{2}{3}\), \(2/3\)\quad\lstinline[style=inline]!\frac{2}{3}!,
      \lstinline[style=inline]!2/3!                                     
  \end{tabular}
\end{center}

\section{Calligraphic letters}
Use as \lstinline[style=inline]!\( \mathcal{A} \)!.
\begin{center}
    \( \mathcal{A}\mathcal{B}\mathcal{C} 
       \mathcal{D}\mathcal{E}\mathcal{F} 
       \mathcal{G}\mathcal{H}\mathcal{I} 
       \mathcal{J}\mathcal{K}\mathcal{L} 
       \mathcal{M}\mathcal{N}\mathcal{O} 
       \mathcal{P}\mathcal{Q}\mathcal{R} 
       \mathcal{S}\mathcal{T}\mathcal{U} 
       \mathcal{V}\mathcal{W}\mathcal{X} 
       \mathcal{Y}\mathcal{Z}  \)
\end{center}


\section{Greek}
\begin{center}
  \begin{tabular}{@{}l@{\hspace{1em}}l@{}}
    $\alpha$\quad\lstinline[style=inline]!\alpha! 
      &$\xi$, $\Xi$\quad\lstinline[style=inline]!\xi!, 
         \lstinline[style=inline]!\Xi!                 \\
    $\beta$\quad\lstinline[style=inline]!\beta!                   
      &o\quad\lstinline[style=inline]!o!                    \\
    $\gamma$, $\Gamma$\quad\lstinline[style=inline]!\gamma!,  
         \lstinline[style=inline]!\Gamma!
      &$\pi$, $\Pi$\quad\lstinline[style=inline]!\pi!, 
        \lstinline[style=inline]!\Pi!                      \\
    $\delta$, $\Delta$\quad\lstinline[style=inline]!\delta!,   
          \lstinline[style=inline]!\Delta! 
       &$\varpi$\quad\lstinline[style=inline]!\varpi!     \\ 
    $\epsilon$\quad\lstinline[style=inline]!\epsilon! 
      &$\rho$\quad\lstinline[style=inline]!\rho! \\
    $\varepsilon$\quad\lstinline[style=inline]!\varepsilon! 
      &$\varrho$\quad\lstinline[style=inline]!\varrho! \\
    $\zeta$\quad\lstinline[style=inline]!\zeta! 
      &$\sigma$, $\Sigma$\quad\lstinline[style=inline]!\sigma!, 
          \lstinline[style=inline]!\Sigma!                   \\
    $\eta$\quad\lstinline[style=inline]!\eta! 
       &$\varsigma$\quad\lstinline[style=inline]!\varsigma!    \\
    $\theta$ $\Theta$\quad\lstinline[style=inline]!\theta!,
          \lstinline[style=inline]!\Theta! 
      &$\tau$\quad \lstinline[style=inline]!\tau!                \\
    $\vartheta$\quad\lstinline[style=inline]!\vartheta!    
      &$\upsilon$, $\Upsilon$\quad\lstinline[style=inline]!\upsilon!,  
            \lstinline[style=inline]!\Upsilon!                     \\
    $\iota$\quad\lstinline[style=inline]!\iota!  
      &$\phi$, $\Phi$\quad\lstinline[style=inline]!\phi!,   
        \lstinline[style=inline]!\Phi!                             \\
    $\kappa$\quad\lstinline[style=inline]!\kappa!
      &$\varphi$\quad\lstinline[style=inline]!\varphi!         \\
    $\lambda$ $\Lambda$\quad\lstinline[style=inline]!\lambda!, 
          \lstinline[style=inline]!\Lambda! 
      &$\chi$\quad\lstinline[style=inline]!\chi!                   \\
    $\mu$\quad\lstinline[style=inline]!\mu!                      
      &$\psi$, $\Psi$\quad\lstinline[style=inline]!\psi!, 
          \lstinline[style=inline]!\Psi!                           \\
    $\nu$\quad\lstinline[style=inline]!\nu!  
      &$\omega$, $\Omega$\quad\lstinline[style=inline]!\omega!, 
            \lstinline[style=inline]!\Omega! 
  \end{tabular}
\end{center}

\section{Sets and logic}
\begin{center} 
  \begin{tabular}{@{}*{2}{l@{\hspace{1em}}}l@{}}
    $\cup$\quad \lstinline[style=inline]!\cup!
      &\(\mathbb{R}\)\quad\lstinline[style=inline]!\mathbb{R}!  
      &$\forall$\quad \lstinline[style=inline]!\forall!        \\
    $\cap$\quad \lstinline[style=inline]!\cap!
      &\(\mathbb{Z}\)\quad\lstinline[style=inline]!\mathbb{Z}!  
      &$\exists$\quad \lstinline[style=inline]!\exists!         \\  
    \(\subset\)\quad\lstinline[style=inline]!\subset!  
      &\(\mathbb{Q}\)\quad\lstinline[style=inline]!\mathbb{Q}!  
      &$\neg$\quad \lstinline[style=inline]!\neg!                \\
    $\subseteq$\quad \lstinline[style=inline]!\subseteq!     
      &\(\mathbb{N}\)\quad\lstinline[style=inline]!\mathbb{N}!  
      &$\vee$\quad \lstinline[style=inline]!\vee!               \\
    $\supset$\quad \lstinline[style=inline]!\supset!       
      &\(\mathbb{C}\)\quad\lstinline[style=inline]!\mathbb{C}!  
      &$\wedge$\quad \lstinline[style=inline]!\wedge!           \\     
    $\supseteq$\quad \lstinline[style=inline]!\supseteq!     
      &$\varnothing$\quad \lstinline[style=inline]!\varnothing!       
      &$\vdash$\quad \lstinline[style=inline]!\vdash!           \\ 
    $\in$\quad \lstinline[style=inline]!\in!            
      &$\emptyset$\quad \lstinline[style=inline]!\emptyset!  
      &$\models$\quad \lstinline[style=inline]!\models!          \\
    $\ni$\quad \lstinline[style=inline]!\ni!     
      &$\aleph$\quad \lstinline[style=inline]!\aleph!  
      &$\Rightarrow$\quad \lstinline[style=inline]!\Rightarrow!   \\    
    $\notin$\quad \lstinline[style=inline]!\notin!     
      &$\setminus$\quad \lstinline[style=inline]!\setminus!        
      &$\nRightarrow$\quad \lstinline[style=inline]!\nRightarrow!    \\  
    $\not\in$\quad \lstinline[style=inline]!\not\in!     
      &$\equiv$\quad \lstinline[style=inline]!\equiv!     
      &  \\     
  \end{tabular}
\end{center}
Negate an operator, as in \(\not\subset\), 
with \lstinline[style=inline]!\not\subset!.
Get the set complement 
$A^{\mathsf{c}}$ with \lstinline[style=inline]!A^{\mathsf{c}}!, 
get $A^{\complement}$ with \lstinline[style=inline]!A^{\complement}!,
or
get $\bar{A}$ with \lstinline[style=inline]!\bar{A}!.



\section{Decorations}
\begin{center}
  \begin{tabular}{@{}*{2}{l@{\hspace{1em}}}l@{}}
    % $f^\prime$\quad\lstinline[style=inline]!f^\prime!
    $f'$\quad\lstinline[style=inline]!f'!
      &$\dot{a}$\quad\lstinline[style=inline]!\dot{a}!      
      &$\tilde{x}$\quad\lstinline[style=inline]!\tilde{x}!      \\
    % $f^{\prime\prime}$\quad\lstinline[style=inline]!f^{\prime\prime}!
    $f''$\quad\lstinline[style=inline]!f''!
      &$\ddot{a}$\quad\lstinline[style=inline]!\ddot{a}!
      &$\bar{x}$\quad\lstinline[style=inline]!\bar{x}!    \\
    $\Sigma^{*}$\quad\lstinline[style=inline]!\Sigma^{*}!
      &$\hat{x}$\quad\lstinline[style=inline]!\hat{x}!
      &$\vec{x}$\quad\lstinline[style=inline]!\vec{x}!
  \end{tabular}
\end{center}
If the decorated letter is $i$ or~$j$ then some decorations need
\lstinline[style=inline]!\imath! or \lstinline[style=inline]!\jmath!, 
as in \lstinline[style=inline]!\vec{\imath}!.
Some authors use boldface for 
vectors: \lstinline[style=inline]!\boldsymbol{x}!.

Entering \lstinline[style=inline]!\overline{x+y}! produces 
$\overline{x+y}$,
and \lstinline[style=inline]!\widehat{x+y}! gives $\widehat{x+y}$.
Comment on an expression as here 
(there is also \lstinline[style=inline]!\overbrace{..}!).
\begin{center}
  $\underbrace{x+y}_{|A|}$\quad\lstinline[style=inline]!\underbrace{x+y}_{|A|}!
\end{center}


\section{Dots}
Use low dots in a list 
$\{0,1,2,\,\ldots\}$, entered as \lstinline[style=inline]!\{0,1,2,\,\ldots\}!. 
(If you use \lstinline[style=inline]!\ldots! in plain text as
\lstinline[style=inline]!London, Paris, \ldots{}\,.! note
the thinspace  \lstinline[style=inline]!\,! before the period.)
Use centered dots in a sum or product 
$1+\cdots+100$, entered as \lstinline[style=inline]!1+\cdots+100!.
You can also get 
vertical dots \lstinline[style=inline]!\vdots!
and diagonal dots \lstinline[style=inline]!\ddots!.

\section{Roman names} Enter \lstinline[style=inline]!\tan(x)!,
with a backslash, instead of \lstinline[style=inline]!tan(x)!.
These get the same treatment.
\begin{center}
  \begin{tabular}{@{}*{3}{l@{\hspace{1.5em}}}l@{}}
    $\sin$\quad \lstinline[style=inline]!\sin!
      &$\sinh$\quad \lstinline[style=inline]!\sinh!     
      &$\arcsin$\quad \lstinline[style=inline]!\arcsin!     \\    
    $\cos$\quad \lstinline[style=inline]!\cos!
      &$\cosh$\quad \lstinline[style=inline]!\cosh!     
      &$\arccos$\quad \lstinline[style=inline]!\arccos!     \\   
    $\tan$\quad \lstinline[style=inline]!\tan!
      &$\tanh$\quad \lstinline[style=inline]!\tanh!      
      &$\arctan$\quad \lstinline[style=inline]!\arctan!     \\
    $\sec$\quad \lstinline[style=inline]!\sec!
      &$\coth$\quad \lstinline[style=inline]!\coth!     
      &$\min$\quad \lstinline[style=inline]!\min!           \\
    $\csc$\quad \lstinline[style=inline]!\csc!  
      &$\det$\quad \lstinline[style=inline]!\det!
      &$\max$\quad \lstinline[style=inline]!\max!           \\
    $\cot$\quad \lstinline[style=inline]!\cot!
      &$\dim$\quad \lstinline[style=inline]!\dim!
      &$\inf$\quad \lstinline[style=inline]!\inf!             \\
    $\exp$\quad \lstinline[style=inline]!\exp! 
      &$\ker$\quad \lstinline[style=inline]!\ker!               
      &$\sup$\quad \lstinline[style=inline]!\sup!             \\
    $\log$\quad \lstinline[style=inline]!\log!
      &$\deg$\quad \lstinline[style=inline]!\deg!
      &$\liminf$\quad \lstinline[style=inline]!\liminf!         \\
    $\ln$\quad \lstinline[style=inline]!\ln!
      &$\arg$\quad \lstinline[style=inline]!\arg!
      &$\limsup$\quad \lstinline[style=inline]!\limsup!         \\
    $\lg$\quad \lstinline[style=inline]!\lg!
      &$\gcd$\quad \lstinline[style=inline]!\gcd!  
      &$\lim$\quad \lstinline[style=inline]!\lim!                \\
    % &$\hom$\quad \lstinline[style=inline]!\hom!                \\
  \end{tabular}
\end{center}


\section{Other symbols}
\begin{center}
  \begin{tabular}{@{}*{2}{l@{\hspace{1.2em}}}l@{}}
    $<$\quad \lstinline[style=inline]!<!  
      &$\angle$\quad \lstinline[style=inline]!\angle!      
      &$\cdot$\quad \lstinline[style=inline]!\cdot!         \\
    $\leq$\quad \lstinline[style=inline]!\leq!
      &$\measuredangle$\quad \lstinline[style=inline]!\measuredangle!      
      &$\pm$\quad \lstinline[style=inline]!\pm!             \\
    $>$\quad \lstinline[style=inline]!>!  
      &$\ell$\quad \lstinline[style=inline]!\ell!  
      &$\mp$\quad \lstinline[style=inline]!\mp!            \\
    $\geq$\quad \lstinline[style=inline]!\geq!
      &$\parallel$\quad \lstinline[style=inline]!\parallel!      
      &$\times$\quad \lstinline[style=inline]!\times!       \\
    $\neq$\quad \lstinline[style=inline]!\neq!
      &$45^{\circ}$\quad \lstinline[style=inline]!45^{\circ}!
      &$\div$\quad \lstinline[style=inline]!\div!            \\
    $\ll$\quad \lstinline[style=inline]!\ll!        
      &$\cong$\quad \lstinline[style=inline]!\cong!        
      &$\ast$\quad \lstinline[style=inline]!\ast!            \\
    $\gg$\quad \lstinline[style=inline]!\gg!      
      &$\ncong$\quad \lstinline[style=inline]!\ncong!        
      &$\mid$\quad \lstinline[style=inline]!\mid!            \\
    $\approx$\quad \lstinline[style=inline]!\approx!      
      &$\sim$\quad \lstinline[style=inline]!\sim!      
      &$\nmid$\quad \lstinline[style=inline]!\nmid!          \\
    $\asymp$\quad \lstinline[style=inline]!\asymp!      
      &$\simeq$\quad \lstinline[style=inline]!\simeq!        
      &$n!$\quad \lstinline[style=inline]+n!+                 \\
    $\equiv$\quad \lstinline[style=inline]!\equiv!      
      &$\nsim$\quad \lstinline[style=inline]!\nsim!      
      &$\partial$\quad \lstinline[style=inline]!\partial!     \\
    $\prec$\quad \lstinline[style=inline]!\prec!        
      &$\oplus$\quad \lstinline[style=inline]!\oplus!
      &$\nabla$\quad \lstinline[style=inline]!\nabla!        \\
    $\preceq$\quad \lstinline[style=inline]!\preceq!      
      &$\ominus$\quad \lstinline[style=inline]!\ominus!   
      &$\hbar$\quad \lstinline[style=inline]!\hbar!        \\
    $\succ$\quad \lstinline[style=inline]!\succ!        
      &$\odot$\quad \lstinline[style=inline]!\odot!      
      &$\circ$\quad \lstinline[style=inline]!\circ!         \\
    $\succeq$\quad \lstinline[style=inline]!\succeq!      
      &$\otimes$\quad \lstinline[style=inline]!\otimes!  
      &$\star$\quad \lstinline[style=inline]!\star!            \\
    $\propto$\quad \lstinline[style=inline]!\propto!      
      &$\oslash$\quad \lstinline[style=inline]!\oslash!    
      &$\surd$\quad \lstinline[style=inline]!\surd!            \\
    $\doteq$\quad \lstinline[style=inline]!\doteq!       
      &$\upharpoonright$\quad \lstinline[style=inline]!\upharpoonright! 
      &$\checkmark$\quad \lstinline[style=inline]!\checkmark!   \\   
  \end{tabular}
\end{center}
Enter \lstinline[style=inline]!a|b! for the divides relation \( a|b \).
Use \lstinline[style=inline]!\mid! as in  
\lstinline[style=inline]!\{a\in S\mid\text{\(a=0\) or \(a\) is odd}\}!
for the set \( \{a\in S \mid\text{\(a=0\) or \(a\) is odd}\} \).

\section{Variable-sized operators}
The summation
\(\sum_{j=0}^3j^2\)\quad\lstinline[style=inline]!\sum_{j=0}^3 j^2!  
and the integral $\int_{x=0}^3x^2\,dx$
\lstinline[style=inline]!\int_{x=0}^3 x^2\,dx!
expand when displayed.
\begin{equation*}
  \sum_{j=0}^3j^2
  \qquad
  \int_{x=0}^3x^2\,dx
\end{equation*}
These do the same.
\begin{center}
  \begin{tabular}{@{}*{2}{l@{\hspace{.35in}}}l@{}}
    $\int$\quad\lstinline[style=inline]!\int!          
      &$\iiint$\quad\lstinline[style=inline]!\iiint!
      &$\bigcup$\quad\lstinline[style=inline]!\bigcup!     \\ 
    $\iint$\quad\lstinline[style=inline]!\iint!
      &$\oint$\quad\lstinline[style=inline]!\oint!       
      &$\bigcap$\quad\lstinline[style=inline]!\bigcap!     \\
      % &$\bigvee$\quad\lstinline[style=inline]!\bigvee!
      % &$\bigwedge$\quad\lstinline[style=inline]!\bigwedge!  \\
  \end{tabular}
\end{center}

\section{Arrows} 
\begin{center}
  \begin{tabular}{@{}l*{1}{@{\hspace{1.5em}}l}@{}}
    $\rightarrow$\quad \lstinline[style=inline]!\rightarrow!,
        \lstinline[style=inline]!\to!
      &$\mapsto$\quad \lstinline[style=inline]!\mapsto!                \\  
    $\nrightarrow$\quad \lstinline[style=inline]!\nrightarrow!     
      &$\longmapsto$\quad \lstinline[style=inline]!\longmapsto!         \\
    $\longrightarrow$\quad \lstinline[style=inline]!\longrightarrow!
      &$\leftarrow$\quad \lstinline[style=inline]!\leftarrow!         \\
    $\Rightarrow$\quad \lstinline[style=inline]!\Rightarrow!
      &$\leftrightarrow$\quad \lstinline[style=inline]!\leftrightarrow! \\
    $\nRightarrow$\quad \lstinline[style=inline]!\nRightarrow!
      &$\downarrow$\quad \lstinline[style=inline]!\downarrow!           \\   
    $\Longrightarrow$\quad \lstinline[style=inline]!\Longrightarrow!
      &$\uparrow$\quad \lstinline[style=inline]!\uparrow!               \\
    $\leadsto$\quad \lstinline[style=inline]!\leadsto!
      &$\updownarrow$\quad \lstinline[style=inline]!\updownarrow!      \\
  \end{tabular}
\end{center}
The right arrows in the first column have matching left arrows, 
such as \lstinline[style=inline]!\nleftarrow!,
and there are some other matches for down arrows, etc.

\section{Fences}
\begin{center}
  \begin{tabular}{@{}*{2}{l@{\hspace{1.5em}}}l@{}}
    $(\;)$\quad \lstinline[style=inline]!()!
      &$\langle\;\rangle$\quad\lstinline[style=inline]!\langle\rangle!  
      &$|\; |$\quad\lstinline[style=inline]!| |!                  \\
    $[\;]$\quad\lstinline[style=inline]![]!                         
      &$\lfloor\;\rfloor$\quad\lstinline[style=inline]!\lfloor\rfloor!
      &$\|\;\|$\quad\lstinline[style=inline]!\| \|!           \\
    $\{\;\}$\quad\lstinline[style=inline]!\{\}!                            
      &$\lceil\;\rceil$\quad\lstinline[style=inline]!\lceil\rceil!     
  \end{tabular}
\end{center}
They will grow with the enclosed formula
using  \lstinline[style=inline]!\left! and \lstinline[style=inline]!\right!.
\begin{center}
  $\displaystyle\left\langle i,2^{2^i}\right\rangle$\quad
    \lstinline[style=inline]!\left\langle i,2^{2^i}\right\rangle!
\end{center}
Every
\lstinline[style=inline]!\left! must match a
\lstinline[style=inline]!\right!
and they must end on the same line in the output.
For a one-sided fence put a period
\lstinline[style=inline]!\left.! or 
\lstinline[style=inline]!\right.! on the other side.%
\begin{center}
  $\displaystyle\left.\frac{df}{dx}\right|_{x_0}$\quad
  \lstinline[style=inline]!\left.\frac{df}{dx}\right|_{x_0}!
\end{center}
Fix the size with  
\lstinline[style=inline]!\big!, 
\lstinline[style=inline]!\Big!, 
\lstinline[style=inline]!\bigg!, or 
\lstinline[style=inline]!\Bigg!.
\begin{center}
  $\displaystyle\Big[\sum_{k=0}^n e^{k^2}\Big]$\quad
    \lstinline[style=inline]!\Big[\sum_{k=0}^n e^{k^2}\Big]!
\end{center}


\section{Arrays, Matrices}
Make an array of mathematical text as you make a 
table of plain text.
\begin{center}
  \vspace*{-\topsep}
  \begin{minipage}{.2\columnwidth}\noindent
  $\displaystyle
  \begin{array}{@{}rcl@{}}
    0 &\leftrightarrow &0 \\
    1 &\leftrightarrow &1 \\
    2 &\leftrightarrow &4 \\
    \vdots &           &\vdots 
  \end{array}
  $
\end{minipage}
\quad
\begin{minipage}{.6\columnwidth}\noindent
\begin{lstlisting}[xleftmargin=0em]
  \begin{array}{rcl}
    0 &\leftrightarrow &0 \\
    1 &\leftrightarrow &1 \\
    2 &\leftrightarrow &4 \\
    \vdots &           &\vdots
  \end{array}
\end{lstlisting}
\end{minipage}
\end{center}
Definition by cases is an array with two columns. 
\begin{center}
  $\displaystyle
  f_n=\begin{cases}
         a              &\text{if \(n=0\)}   \\
         r\cdot f_{n-1}  &\text{else}   
       \end{cases}
  $
\quad
\begin{minipage}{.525\columnwidth}\noindent
\begin{lstlisting}[xleftmargin=-1em]
  f_n=
  \begin{cases}
   a     &\text{if \(n=0\)} \\
   r\cdot f_{n-1} &\text{else}   
  \end{cases}
\end{lstlisting}
\end{minipage}
\end{center}
A matrix is another array variant.
With this abbreviation you need not specify column alignments.
\begin{center}
  $\displaystyle
  \begin{pmatrix}
    a  &b \\
    c  &d
  \end{pmatrix}
  $
  \quad
\begin{minipage}{.525\columnwidth}\noindent
\begin{lstlisting}[xleftmargin=0em]
  \begin{pmatrix}
    a  &b \\
    c  &d
  \end{pmatrix}
\end{lstlisting}
\end{minipage}
\end{center}
For the determinant use \lstinline[style=inline]!|A|! inline
and \lstinline[style=inline]!vmatrix! in display.

\section{Spacing in mathematics}
\begin{center}
  \begin{tabular}{@{}*{1}{l@{\hspace{2.25em}}}l@{}}
    \(\rightarrow\,\leftarrow\)\quad\lstinline[style=inline]!\,!
      &\(\rightarrow\quad\leftarrow\)\quad\lstinline[style=inline]!\quad!   \\
    \(\rightarrow\:\leftarrow\)\quad\lstinline[style=inline]!\:!
      &\(\rightarrow\qquad\leftarrow\)\quad\lstinline[style=inline]!\qquad! \\
    \(\rightarrow\;\leftarrow\)\quad\lstinline[style=inline]!\;!
      &\(\rightarrow\!\leftarrow\)\quad\lstinline[style=inline]+\!+           
  \end{tabular}
\end{center}
The left column spaces are in ratio \(3\mathbin{:}4\mathbin{:}5\).
The last in the right column is a negative space, 
opposite to \lstinline[style=inline]!\,!.
Get arbitrary space as in 
% \(\rightarrow\hspace*{0.5cm}\leftarrow\)
\lstinline[style=inline]!\hspace{0.5cm}!.


\section{Displayed equations}
Put equations on a separate line with the
\lstinline[style=inline]!equation*! environment.
\begin{center}
\vspace*{-\topsep}
  \begin{minipage}{0.25\linewidth}\vspace*{-\abovedisplayskip}
    \begin{equation*}
      S=k\log W
    \end{equation*}
  \end{minipage}
  \quad
  \begin{minipage}{0.65\linewidth}
\begin{lstlisting}[xleftmargin=0ex]
\begin{equation*}
  S=k\log W
\end{equation*}
\end{lstlisting}
  \end{minipage}
\end{center}
You can break into multiple lines. 
\begin{center}
\vspace*{-\topsep}
  \begin{minipage}{0.35\linewidth}\vspace*{-\abovedisplayskip}
    \begin{multline*}
      \sin (x)=x-\frac{x^3}{3!} \\
          +\frac{x^5}{5!}-\cdots
    \end{multline*}
  \end{minipage}
  \quad
  \begin{minipage}{0.55\linewidth}
\begin{lstlisting}[xleftmargin=0ex]
\begin{multline*}
  \sin (x)=x-\frac{x^3}{3!} \\
      +\frac{x^5}{5!}-\cdots
\end{multline*}
\end{lstlisting}
  \end{minipage}
\end{center}
Align using the \lstinline[style=inline]!align*! environment
\vspace*{-\topsep}
\begin{center}
  \begin{minipage}[c]{0.25\linewidth}\vspace*{-\abovedisplayskip}
    \begin{align*}
      \nabla\cdot\boldsymbol{D} &= \rho \\
      \nabla\cdot\boldsymbol{B} &= 0    
    \end{align*}
  \end{minipage}
  \quad
  \begin{minipage}[c]{0.65\linewidth}
\begin{lstlisting}[xleftmargin=0ex,boxpos=c]
\begin{align*}
  \nabla\cdot\boldsymbol{D} &= \rho \\
  \nabla\cdot\boldsymbol{B} &= 0    
\end{align*}
\end{lstlisting}
  \end{minipage}
\end{center}
(you can have an empty left or right side of the alignment).
For each environment, get a numbered version by 
dropping the asterisk from the name.


\section{Calculus examples}
The last three here are display style.
\begin{center} \small \def\arraystretch{2}
  \vspace*{-\topsep}\vspace*{-0.25cm}
  \begin{tabular}{@{}r@{\hspace{1em}}l@{}}
    \( f\colon\mathbb{R}\to\mathbb{R} \)  
      &\lstinline[style=inline]!f\colon\mathbb{R}\to\mathbb{R}!      \\
    \( 9.8~\text{m}/\text{s}^2 \)  
      &\lstinline[style=inline]!9.8~\text{m}/\text{s}^2!      \\
    $\displaystyle
     \lim_{h\to 0}\frac{f(x+h)-f(x)}{h}
    $
    &\lstinline[style=inline]!\lim_{h\to 0}\frac{f(x+h)-f(x)}{h}!   \\
    $\displaystyle
     \int x^2\,dx=x^3/3+C
    $
    &\lstinline[style=inline]!\int x^2\,dx=x^3/3+C!             \\
    % $\displaystyle
    %  \frac{dy}{dx}=\frac{dy}{du}\frac{du}{dx}
    % $
    % &\lstinline[style=inline]!\frac{dy}{dx}=\frac{dy}{du}\frac{du}{dx}! \\
    $\displaystyle
     \nabla=\boldsymbol{i}\frac{d}{dx}
             +\boldsymbol{j}\frac{d}{dy}
             +\boldsymbol{k}\frac{d}{dz}
    $
    &\lstinline[style=inline]!\nabla=\boldsymbol{i}\frac{d}{dx}+!\,\(\cdots\)\\
  \end{tabular}
\end{center}

\section{Discrete mathematics examples}
There are four modulo forms: 
\(m\bmod n\) is from \lstinline[style=inline]!m\bmod n!,
and \(a\equiv b\pmod m\) is from \lstinline[style=inline]!a\equiv b\pmod m!,
and \(a\equiv b\mod m\) is from \lstinline[style=inline]!a\equiv b\mod m!,
and \(a\equiv b\pod m\) is from \lstinline[style=inline]!a\equiv b\pod m!.

For combinations the binomial symbol $\binom{n}{k}$ is from
\lstinline[style=inline]!\binom{n}{k}!.
This resizes to be bigger in a display
(to require the display version use \lstinline[style=inline]!\dbinom{n}{k}!
and for the inline version use \lstinline[style=inline]!\tbinom{n}{k}!).

For permutations use 
\( n^{\underline{r}} \) from \lstinline[style=inline]!n^{\underline{r}}! 
(some authors use \( P(n,r) \), 
or \( {}_nP_r \) from  \lstinline[style=inline]!{}_nP_r!).


\section{Statistics examples}
\begin{center} \small\def\arraystretch{1.5}
  \vspace*{-\topsep}\vspace*{-0.15cm}
  \begin{tabular}{@{}r@{\hspace{1em}}l@{}}
    $
     \sigma^2=\sqrt{\sum (x_i-\mu)^2/N}
    $
    &\lstinline[style=inline]!\sigma^2=\sqrt{\sum (x_i-\mu)^2/N}!   \\
    $
     E(X)=\mu_X=\sum (x_i-P(x_i))
    $
    &\lstinline[style=inline]! E(X)=\mu_X=\sum (x_i-P(x_i))!             \\
  \end{tabular}
\end{center}
The probability density of the normal distribution 
\begin{equation*}
     \frac{1}{\sqrt{2\sigma^2\pi}}\,e^{-\frac{(x-\mu)^2}{2\sigma^2}}
\end{equation*}
comes from this.
\begin{lstlisting}
   \frac{1}{sqrt{2\sigma^2\pi}}
      \,e^{-\frac{(x-\mu)^2}{2\sigma^2}}!.
\end{lstlisting}


\section{For more} 
See also the 
Comprehensive \LaTeX{} Symbols 
List at \url{mirror.ctan.org/info/symbols/comprehensive}
and De\TeX ify at \url{detexify.kirelabs.org/classify.html}.


\vfill
\noindent\parbox{\columnwidth}{\small%
{\color{headcolor}\hrulefill\\
\vspace*{0ex}
Jim Hef{}feron, Saint Michael's College, VT USA\hfill 2017-Jan-10
}}
\end{document}
